\documentclass[11pt]{article}
\usepackage[utf8]{inputenc}
\usepackage{amsmath}
\usepackage{amssymb}
\usepackage{geometry}
\geometry{a4paper, margin=1in}

\title{Procedure for Satellite Crossing and Visibility Analysis}
\author{Gilari Ramachandran Karthi}
\date{25-Feb-2026}

\begin{document}
	
	\maketitle
	
	\section{Coordinate Reference Frame}
	
	All computations are performed in the \textbf{True Equator Mean Equinox (TEME)} inertial frame to ensure consistency between:
	
	\begin{itemize}
		\item Target propagation using SGP4 from TLE,
		\item Tracker propagation using Keplerian elements transformed to TEME.
	\end{itemize}
	
	The state vectors are defined as:
	
	\begin{equation}
		\mathbf{r} = [x, y, z]^T, \quad 
		\mathbf{v} = [\dot{x}, \dot{y}, \dot{z}]^T
	\end{equation}
	
	\section{Relative Geometry and FOV Crossing}
	
	\subsection{Relative Position Vector}
	
	The relative position vector from Tracker to Target is:
	
	\begin{equation}
		\mathbf{d}(t) = \mathbf{r}_{tgt}(t) - \mathbf{r}_{trk}(t)
	\end{equation}
	
	The relative range is:
	
	\begin{equation}
		\rho(t) = \|\mathbf{d}(t)\|
	\end{equation}
	
	\subsection{Field-of-View (FOV) Condition}
	
	The camera boresight is aligned with the Tracker velocity vector $\mathbf{v}_{trk}$.
	
	The angular separation between the boresight and the target direction is:
	
	\begin{equation}
		\cos \theta = 
		\frac{\mathbf{d} \cdot \mathbf{v}_{trk}}
		{\|\mathbf{d}\| \, \|\mathbf{v}_{trk}\|}
	\end{equation}
	
	A \textbf{crossing event} is detected when:
	
	\begin{equation}
		\theta \leq \frac{FOV}{2} = 15^\circ
	\end{equation}
	
	\section{Visibility Conditions}
	
	A target is considered \textbf{visible/detectable} only if all physical constraints are satisfied simultaneously.
	
	\subsection{Range Constraint}
	
	\begin{equation}
		\rho(t) < R_{max}
	\end{equation}
	
	where:
	
	\begin{equation}
		R_{max} = 1000 \text{ km}
	\end{equation}
	
	\subsection{Illumination (Sunlight) Condition}
	
	Let:
	
	\begin{itemize}
		\item $\mathbf{r}_{tgt}$ = Earth $\rightarrow$ Target
		\item $\mathbf{r}_{sun}$ = Earth $\rightarrow$ Sun
		\item $R_E$ = Earth radius
	\end{itemize}
	
	Define:
	
	\begin{equation}
		f_{e,1} := (\mathbf{r}_{tgt} \cdot \mathbf{r}_{sun} > 0)
	\end{equation}
	
	\begin{equation}
		f_{e,2} := 
		\left[
		\frac{(\mathbf{r}_{tgt} \cdot \mathbf{r}_{sun})^2}
		{\|\mathbf{r}_{sun}\|^2}
		- \|\mathbf{r}_{tgt}\|^2
		+ R_E^2 < 0
		\right]
	\end{equation}
	
	The target is sunlit if:
	
	\begin{equation}
		f_e = f_{e,1} \wedge f_{e,2}
	\end{equation}
	
	This represents a cylindrical Earth shadow model.
	
	\subsection{Above-Horizon (Line-of-Sight) Condition}
	
	To ensure Earth does not block the line-of-sight between Tracker and Target:
	
	Let:
	
	\begin{equation}
		\mathbf{d} = \mathbf{r}_{tgt} - \mathbf{r}_{trk}
	\end{equation}
	
	\begin{equation}
		r_{o,e} = \sqrt{\|\mathbf{r}_{trk}\|^2 - R_E^2}
	\end{equation}
	
	The above-horizon condition is:
	
	\begin{equation}
		f_o := 
		\left(
		\|\mathbf{d}\| \, r_{o,e} + \mathbf{d} \cdot \mathbf{r}_{trk} > 0
		\right)
	\end{equation}
	
	This ensures that the Target is not geometrically occluded by Earth.
	
	\section{Final Detection Condition}
	
	A target is considered \textbf{detectable} if:
	
	\begin{equation}
		f_{vis} = 
		(\theta \leq 15^\circ)
		\wedge
		(\rho < 1000)
		\wedge
		f_e
		\wedge
		f_o
	\end{equation}
	
	\section{Time Propagation and Event Grouping}
	
	The simulation is performed over:
	
	\begin{equation}
		T = 86400 \text{ s}
	\end{equation}
	
	with discrete time steps:
	
	\begin{equation}
		t_{n+1} = t_n + \Delta t
	\end{equation}
	
	where $\Delta t = 5$ s.
	
	Crossing timestamps are recorded whenever the FOV condition is satisfied. Consecutive timestamps separated by exactly $\Delta t$ are grouped into continuous intervals:
	
	\begin{equation}
		[t_{start}, t_{end}]
	\end{equation}
	
	representing one geometric crossing event.
	
\end{document}